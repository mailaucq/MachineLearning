\documentclass[preview,border=4mm,convert={density=600,outext=.png}]{standalone}
\usepackage{tikz}
\begin{document}
\pagestyle{empty}
\def\layersep{3cm}
\def\nodeinlayersep{2cm}

\begin{tikzpicture}[
   shorten >=1pt,->,
   draw=black!50,
    node distance=\layersep,
    every pin edge/.style={<-,shorten <=1pt},
    neuron/.style={circle,fill=black!25,minimum size=17pt,inner sep=0pt},
    input neuron/.style={neuron, fill=green!50,},
    output neuron/.style={neuron, fill=red!50},
    hidden neuron/.style={neuron, fill=blue!50},
    annot/.style={text width=4em, text centered},
    bias/.style={neuron, fill=yellow!50},%<-- added %%%
]
    % Draw the input layer nodes
    \foreach \name / \y in {1,...,2}
    \node[input neuron, pin=left:x\y] (I-\name) at (0,-\y-1.5) {I(x)};  
    % set number of hidden layers
    \newcommand\Nhidden{1}
    \newcommand\Nlasthidden{2}
    % Draw the hidden layer nodes
    \foreach \N in {0,...,\Nhidden} {
       \foreach \y in {0,...,2} { % <-- added 0 instead of 1 %%%%%
     \ifnum \y=4
     \ifnum \N>0 
       \node at (\N*\layersep,-\y*\nodeinlayersep) {$\vdots$};
       \else\fi 
     \else
         \ifnum \y=0 
         \ifnum \N<3
           \node[bias] (H\N-\y) at (\N*\layersep,-\y*\nodeinlayersep ) {1};
           \else\fi
         \else
            \ifnum \N>0
           \node[hidden neuron] (H\N-\y) at (\N*\layersep,-\y*\nodeinlayersep ) {f(net)};
               \else\fi 
         \fi 
         \fi
       }
       \ifnum \N>0
    \node[annot,above of=H\N-1, node distance=1cm,yshift=2cm] (hl\N) {Hidden layer \N}; % <- added yshift=2cm %%%%%%%%%%%%
    \else\fi 
    }
    % Draw the output layer node
    \node[output neuron,pin={[pin edge={->}]right: $\hat{y} = [0, 1]$}, right of=H\Nhidden-2] (O)  at (\Nhidden*\layersep,-\Nlasthidden-1){f(net)}; 
    % Connect every node in the input layer with every node in the
    % hidden layer.
    \foreach \source in {1,...,2}
        \foreach \dest in {1,...,2} {
          % \path[yellow] (H-0) edge (H1-\dest);
          \path[dashed,orange] (H0-0) edge node[near start] {$\theta_{\dest}^h$}  (H1-\dest); 
            \path[blue] (I-\source) edge node[near start] {$w_{\dest\source}^h$} (H1-\dest);};

    % connect all hidden stuff
    \ifnum \Nhidden>1
    \foreach [remember=\N as \lastN (initially 1)] \N in {2,...,\Nhidden}
       \foreach \source in {0,...,3,5} 
           \foreach \dest in {1,...,3,5}{

               \ifnum \source=0 
           \path[dashed,orange](H\lastN-\source) edge (H\N-\dest);
              \else 
              \path[green] (H\lastN-\source) edge (H\N-\dest);
              \fi 
              }; 
    \fi 
    % Connect every node in the hidden layer with the output layer
    \foreach \source in {1,...,2}
    \foreach \dest in {1}{
    \path[red] (H\Nhidden-\source) edge node[near start] {$w_{\dest\source}^\sigma$}  (O);
    \path[dashed,orange] (H1-0) edge node[near start] {$\theta_{\dest}^\sigma$}  (O);}; 
    % Annotate the layers
    \node[annot,left of=hl1] {Input layer};
    \node[annot,right of=hl\Nhidden] {Output layer};  
\end{tikzpicture}
% End of code


% End of code
\end{document}


